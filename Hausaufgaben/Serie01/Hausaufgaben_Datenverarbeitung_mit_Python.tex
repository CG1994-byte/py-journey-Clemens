\documentclass[11pt, a4paper]{article}

% --- Standard Pakete ---
\usepackage[utf8]{inputenc}
\usepackage[T1]{fontenc}
\usepackage[ngerman]{babel}
\usepackage{geometry} % Für Seitenränder etc.
\geometry{a4paper, margin=2.5cm} % Standardseitenränder
\usepackage{parskip} % Für Absätze ohne Einzug, stattdessen mit Abstand (optional, kann entfernt werden, wenn traditionelles Layout gewünscht)
\usepackage{enumitem} % Für bessere Kontrolle über Listen

% --- Grafik und Farbe ---
\usepackage{graphicx}
\usepackage[svgnames]{xcolor} % Bietet mehr Farbnamen
\usepackage{fontawesome5}      % Für Icons wie \faFileAlt (ersetzt fontawesome, da moderner)

% --- Listings für Code ---
\usepackage{listings}

% --- TColorBox für Blöcke ---
\usepackage{tcolorbox}
\tcbuselibrary{skins, breakable} % breakable für Seitenumbrüche in Boxen

% --- Hyperref für Links ---
\usepackage{hyperref}
\hypersetup{
    colorlinks=true,
    linkcolor=MidnightBlue, % Dunkelblau für interne Links
    urlcolor=DarkSlateGray,  % Dunkelgrau für URLs
    citecolor=DarkGreen      % Falls Zitate verwendet werden
}

% --- Custom colors (aus deiner Präambel übernommen/angepasst) ---
\definecolor{codeblue}{rgb}{0.1, 0.1, 0.8}
\definecolor{codegray}{rgb}{0.5, 0.5, 0.5}
\definecolor{codepurple}{rgb}{0.6, 0.1, 0.6}
\definecolor{backcolor}{rgb}{0.98, 0.98, 0.97} % Hellerer Hintergrund für Code
\definecolor{MyRed}{HTML}{D72638} % Für Alertboxen
\definecolor{MyBlue}{HTML}{4A90E2} % Für Infoboxen

% --- Listings Stil Definition ---
\lstdefinestyle{mystyle}{
    language=Python,
    backgroundcolor=\color{backcolor},
    commentstyle=\itshape\color{codegray},
    keywordstyle=\bfseries\color{codeblue},
    numberstyle=\tiny\color{codegray},
    stringstyle=\color{codepurple},
    basicstyle=\ttfamily\footnotesize, % Etwas größer als \tiny für bessere Lesbarkeit im Artikel
    breakatwhitespace=false,
    breaklines=true,
    captionpos=b, % Bildunterschrift unten
    keepspaces=true,
    numbers=left,
    numbersep=5pt,
    showspaces=false,
    showstringspaces=false,
    showtabs=false,
    tabsize=2,
    literate={ä}{{\\texttt{"}a}}1 {ö}{{\\texttt{"}o}}1 {ü}{{\\texttt{"}u}}1 {Ä}{{\\texttt{"}A}}1 {Ö}{{\\texttt{"}O}}1 {Ü}{{\\texttt{"}U}}1 {ß}{{\ss}}1
}
\lstset{style=mystyle}

% --- TColorBox Definitionen (angepasst für Artikel) ---
\newtcolorbox{alertbox}[2][]{%
    enhanced,
    title=#2,
    attach boxed title to top left={xshift=3mm, yshift=-2.5mm},
    colback=MyRed!10,
    colframe=MyRed!90!black,
    fonttitle=\bfseries\sffamily,
    coltitle=white,
    colbacktitle=MyRed!90!black,
    boxed title style={arc=2mm,outer arc=2mm},
    sharp corners=all,
    boxrule=1pt,
    arc=0mm,
    breakable,
    #1
}

\newtcolorbox{infoblock}[2][]{%
    enhanced,
    title=#2,
    attach boxed title to top left={xshift=3mm, yshift=-2.5mm},
    colback=MyBlue!10,
    colframe=MyBlue!80!black,
    fonttitle=\bfseries\sffamily,
    coltitle=white,
    colbacktitle=MyBlue!80!black,
    boxed title style={arc=2mm,outer arc=2mm},
    sharp corners=all,
    boxrule=1pt,
    arc=0mm,
    breakable,
    #1
}

% --- Dokument Metadaten ---
\title{Hausaufgaben: Datenverarbeitung mit Python}
\author{Jowan Sulaiman \\ ABS IT}
\date{\today}

\begin{document}
\maketitle
\thispagestyle{empty} % Keine Seitenzahl auf der Titelseite
\clearpage

\clearpage
\setcounter{page}{1} % Seitenummerierung ab hier neu starten

% --- Hausaufgabe: Python File Handling (Anfänger) ---
\section*{Hausaufgabe 1: Grundlagen der Dateibearbeitung in Python \faFolderOpen}

\subsection*{Theoriefragen}
Beantworte die folgenden Fragen in eigenen Worten. Schreibe deine Antworten in eine separate Text- oder Markdown-Datei.

\begin{enumerate}
    \item Was ist der Unterschied zwischen den Dateiöffnungsmodi \texttt{\texttt{"}r\texttt{"}}, \texttt{\texttt{"}w\texttt{"}} und \texttt{\texttt{"}a\texttt{"}} in Python? Erkläre kurz, was passiert, wenn du eine Datei in jedem dieser Modi öffnest.
    \item Warum ist die Zeile \texttt{with open(\texttt{"}meinedatei.txt\texttt{"}, \texttt{"}r\texttt{"}) as f:} eine gute Methode, um Dateien in Python zu öffnen und zu lesen? Nenne einen wichtigen Vorteil.
    \item Stell dir vor, du hast eine sehr große Textdatei (z.B. ein Logbuch mit vielen Einträgen). Wäre es besser, die Datei Zeile für Zeile zu lesen oder den gesamten Inhalt auf einmal? Begründe kurz.
    \item Das \texttt{os}-Modul in Python ist nützlich für Dateioperationen. Nenne eine Funktion aus dem \texttt{os}-Modul und beschreibe, was sie tut.
\end{enumerate}

\subsection*{Praktische Aufgaben \faLaptopCode}

\subsubsection*{Aufgabe 1.1: Mein einfaches Notizbuch}
Schreibe ein Python-Skript, das als einfaches Notizbuch dient.

\begin{itemize}
    \item Das Programm soll den Benutzer fragen, ob er eine neue Notiz hinzufügen oder alle Notizen anzeigen möchte.
    \item \texttt{Notiz hinzufügen:} Wenn der Benutzer eine Notiz hinzufügen möchte, soll das Programm ihn nach der Notiz fragen (eine einzelne Zeile Text genügt). Diese Notiz soll dann in einer Datei namens \texttt{meine\_notizen.txt} gespeichert werden. Jede neue Notiz soll in einer neuen Zeile angehängt werden.
    \item \textbf{Notizen anzeigen:} Wenn der Benutzer die Notizen anzeigen möchte, soll das Programm den gesamten Inhalt der Datei \texttt{meine\_notizen.txt} lesen und auf der Konsole ausgeben.
    \item \textbf{Beispiel-Interaktion:}
\begin{verbatim}
Was möchtest du tun? (schreiben / lesen / beenden): schreiben
Gib deine Notiz ein: Heute Python gelernt!
Notiz gespeichert.

Was möchtest du tun? (schreiben / lesen / beenden): lesen
--- Meine Notizen ---
Heute Python gelernt!
--- Ende der Notizen ---

Was möchtest du tun? (schreiben / lesen / beenden): beenden
\end{verbatim}
\end{itemize}
\begin{infoblock}{Tipp}
Verwende den Modus \texttt{\texttt{"}a\texttt{"}} zum Anhängen an die Datei und \texttt{\texttt{"}r\texttt{"}} zum Lesen. Denke an den \texttt{with}-Block!
\end{infoblock}

\subsubsection*{Aufgabe 1.2: Datei-Inspektor}
Schreibe ein Python-Skript, das den Benutzer nach einem Dateinamen fragt und dann versucht, den Inhalt dieser Datei anzuzeigen.

\begin{itemize}
    \item Frage den Benutzer: \texttt{"}Welche Datei möchtest du anzeigen?\texttt{"}
    \item Lies den Dateinamen ein.
    \item Versuche, die Datei zu öffnen und ihren gesamten Inhalt auf der Konsole auszugeben.
    \item \textbf{Fehlerbehandlung:} Wenn die Datei nicht gefunden wird, soll das Programm nicht abstürzen, sondern eine freundliche Meldung ausgeben, z.B. \texttt{"}Datei '[Dateiname]' nicht gefunden.\texttt{"}
\end{itemize}
\begin{infoblock}{Tipp}
Verwende einen \texttt{try-except}-Block, um den \texttt{FileNotFoundError} abzufangen.
\end{infoblock}

\subsubsection*{Aufgabe 1.3: Ordner-Inhalt auflisten}
Schreibe ein Python-Skript, das alle Dateien und Ordner in einem vom Benutzer angegebenen Verzeichnis auflistet.

\begin{itemize}
    \item Frage den Benutzer: \texttt{"}Welchen Ordner möchtest du auflisten?\texttt{"}
    \item Lies den Pfad zum Ordner ein.
    \item Verwende das \texttt{os}-Modul (insbesondere \texttt{os.listdir()}), um alle Einträge (Dateien und Unterordner) in diesem Verzeichnis aufzulisten und ihre Namen auf der Konsole auszugeben.
    \item \textbf{Fehlerbehandlung:} Gib eine Meldung aus, falls der angegebene Pfad kein gültiges Verzeichnis ist.
\end{itemize}

\clearpage
% --- Hausaufgabe: JSON in Python (Anfänger) ---
\section*{Hausaufgabe 2: Arbeiten mit JSON-Daten \faFileCode}

\subsection*{Theoriefragen}
Beantworte die folgenden Fragen in eigenen Worten.

\begin{enumerate}
    \item Was ist JSON? Gib ein einfaches Beispiel, wofür man JSON verwenden könnte.
    \item Nenne drei verschiedene Datentypen, die in JSON verwendet werden können (z.B. String, Zahl, ...). Gib für jeden dieser Typen an, welchem Python-Datentyp er entspricht.
    \item Erkläre kurz den Unterschied zwischen \texttt{json.dumps()} und \texttt{json.dump()} in Python. (Ein Satz pro Funktion genügt).
    \item Wenn du eine Python-Struktur (z.B. ein Dictionary) mit \texttt{json.dumps(daten, indent=4)} in einen JSON-String umwandelst, was bewirkt der Parameter \texttt{indent=4}?
\end{enumerate}

\subsection*{Praktische Aufgaben \faLaptopCode}

\subsubsection*{Aufgabe 2.1: Mein Lieblingsessen speichern}
Schreibe ein Python-Skript, das Informationen über dein Lieblingsessen in einer JSON-Datei speichert und wieder ausliest.

\begin{itemize}
    \item Erstelle ein Python-Dictionary, das dein Lieblingsessen beschreibt. Es sollte mindestens die Schlüssel \texttt{\texttt{"}name\texttt{"}} (z.B. \texttt{"}Pizza Margherita\texttt{"}), \texttt{\texttt{"}kategorie\texttt{"}} (z.B. \texttt{"}Italienisch\texttt{"}) und \texttt{\texttt{"}bewertung\texttt{"}} (eine Zahl von 1 bis 5) enthalten.
    \item Speichere dieses Dictionary als schön formatierte JSON-Datei unter dem Namen \texttt{lieblingsessen.json}. Verwende eine Einrückung von 2 Leerzeichen.
    \item Schreibe dann Code, der die Datei \texttt{lieblingsessen.json} wieder einliest und den Namen deines Lieblingsessens auf der Konsole ausgibt.
\end{itemize}
\begin{infoblock}{Tipp}
Du benötigst \texttt{json.dump()} zum Speichern und \texttt{json.load()} zum Laden.
\end{infoblock}

\subsubsection*{Aufgabe 2.2: Einfache Kontaktliste}
Du möchtest eine kleine Liste von Kontakten verwalten.

\begin{itemize}
    \item Erstelle eine Python-Liste, die zwei oder drei Kontakte enthält. Jeder Kontakt soll ein Dictionary sein mit den Schlüsseln \texttt{\texttt{"}name\texttt{"}} und \texttt{\texttt{"}telefonnummer\texttt{"}}.
        \textit{Beispiel:}
\begin{lstlisting}
kontakte = [
    {\texttt{"}name\texttt{"}: \texttt{"}Max Mustermann\texttt{"}, \texttt{"}telefonnummer\texttt{"}: \texttt{"}0123-45678\texttt{"}},
    {\texttt{"}name\texttt{"}: \texttt{"}Erika Musterfrau\texttt{"}, \texttt{"}telefonnummer\texttt{"}: \texttt{"}0987-65432\texttt{"}}
]
\end{lstlisting}
    \item Speichere diese Liste von Kontakten in einer JSON-Datei namens \texttt{kontakte.json}.
    \item Schreibe Code, der \texttt{kontakte.json} einliest und alle Namen und Telefonnummern formatiert auf der Konsole ausgibt.
\end{itemize}

\subsubsection*{Aufgabe 2.3: JSON-String verarbeiten}
Gegeben ist der folgende Python-String, der JSON-Daten enthält:
\begin{lstlisting}
json_text = '{\texttt{"}buch_titel\texttt{"}: \texttt{"}Python für Anfänger\texttt{"}, \texttt{"}kapitel\texttt{"}: 10, \texttt{"}verfuegbar\texttt{"}: true}'
\end{lstlisting}
\begin{itemize}
    \item Wandle diesen JSON-String in ein Python-Dictionary um.
    \item Gib den Wert des Schlüssels \texttt{"}\texttt{buch\_titel}\texttt{"} aus.
    \item Gib den Wert des Schlüssels \texttt{\texttt{"}kapitel\texttt{"}} aus.
\end{itemize}
\begin{infoblock}{Tipp}
Verwende \texttt{json.loads()} (achte auf das 's' am Ende!).
\end{infoblock}

\clearpage
% --- Hausaufgabe: RegEx in Python (Anfänger) ---
\section*{Hausaufgabe 3: Einführung in Reguläre Ausdrücke (RegEx) \faSearchPlus}

\subsection*{Theoriefragen}
Beantworte die folgenden Fragen in eigenen Worten.

\begin{enumerate}
    \item Was ist ein Regulärer Ausdruck (kurz RegEx)? Versuche, es mit einem einfachen Beispiel zu erklären.
    \item Erkläre kurz die Bedeutung der folgenden einfachen RegEx-Zeichen/Metazeichen:
        \begin{itemize}
            \item \texttt{.} (der Punkt)
            \item \texttt{*} (der Stern)
            \item \texttt{\textbackslash{}d} (Backslash d)
            \item \texttt{[]} (eckige Klammern)
        \end{itemize}
    \item Was ist der Unterschied zwischen \texttt{re.search()} und \texttt{re.findall()} aus dem Python \texttt{re}-Modul?
    \item Warum ist es oft eine gute Idee, Reguläre Ausdrücke in Python als \texttt{"}Raw Strings\texttt{"} zu schreiben (z.B. \texttt{r\texttt{"}mein\_muster\texttt{"}})?
\end{enumerate}

\subsection*{Praktische Aufgaben \faLaptopCode}

\begin{alertbox}{Hinweis zu RegEx}
Reguläre Ausdrücke können anfangs knifflig sein. Nutze Online-Tools wie [regex101.com](https://regex101.com/) (stelle die \texttt{"}Flavor\texttt{"} auf Python ein), um deine Muster zu testen und besser zu verstehen!
\end{alertbox}

\subsubsection*{Aufgabe 3.1: Finde die Zahlen}
Gegeben ist der folgende Text:
\begin{verbatim}
"Das Produkt A1 kostet 25 Euro, Produkt B22 kostet 199 Euro und Artikel C333 
kostet 12 Euro."
\end{verbatim}
\begin{itemize}
    \item Schreibe ein Python-Skript, das alle Preisangaben (nur die Zahlen, z.B. 25, 199, 12) aus diesem Text extrahiert und als Liste ausgibt.
\end{itemize}
\begin{infoblock}{Tipp}
Du suchst nach einer oder mehreren Ziffern. \texttt{\textbackslash{}d+} könnte hier nützlich sein. Verwende \texttt{re.findall()}.
\end{infoblock}

\subsubsection*{Aufgabe 3.2: Einfache \texttt{"}Hallo\texttt{"}-Suche}
Schreibe ein Python-Skript, das prüft, ob das Wort \texttt{"}Hallo\texttt{"} (genau so geschrieben, mit großem 'H') in einem vom Benutzer eingegebenen Text vorkommt.

\begin{itemize}
    \item Frage den Benutzer nach einem Satz.
    \item Verwende \texttt{re.search()}, um zu prüfen, ob \texttt{"}Hallo\texttt{"} im Satz enthalten ist.
    \item Gib aus: \texttt{"}Hallo gefunden!\texttt{"} oder \texttt{"}Hallo nicht gefunden.\texttt{"}
\end{itemize}

\subsubsection*{Aufgabe 3.3: Wörter mit 'test' finden}
Gegeben ist der Satz:
\begin{verbatim}
"Das ist ein Test. Dieser Testfall ist wichtig. Ich teste gerne."
\end{verbatim}
\begin{itemize}
    \item Schreibe ein Python-Skript, das alle Wörter findet, die die Zeichenkette \texttt{"}test\texttt{"} (unabhängig von Groß-/Kleinschreibung) enthalten.
    \item Gib die gefundenen Wörter als Liste aus. (Erwartete Ausgabe könnte sein: ['Test', 'Testfall', 'teste'])
\end{itemize}
\begin{infoblock}{Tipp}
Suche nach Wörtern, die \texttt{"}test\texttt{"} enthalten. \texttt{\textbackslash{}w*test\textbackslash{}w*} könnte ein Ansatz sein. Um Groß-/Kleinschreibung zu ignorieren, kannst du das Flag \texttt{re.IGNORECASE} verwenden (z.B. \texttt{re.findall(muster, text, re.IGNORECASE)}).
\end{infoblock}

\end{document}